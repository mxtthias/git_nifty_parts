\documentclass[11pt,xetex]{beamer}

\usetheme[numbering=none]{metropolis}

\usepackage{hyperref}
\usepackage{minted}
\AtBeginEnvironment{minted}{\fontsize{10}{10}\selectfont}
\usemintedstyle{trac}

\setsansfont{League Spartan}

\definecolor{normalTextColor}{HTML}{303030}
\definecolor{alertedTextColor}{HTML}{EF746F}
\definecolor{exampleTextColor}{HTML}{1693A5}

\setbeamercolor{normal text}{fg=normalTextColor, bg=black!2}
\setbeamercolor{alerted text}{fg=alertedTextColor, bg=black!2}
\setbeamercolor{example text}{fg=exampleTextColor, bg=black!2}

\title{Git: (some of) the nifty parts}
\date{\today}
\author{Matthias Nilsson}
%\institute{}

\begin{document}

\maketitle

\section*{Introduction}

\begin{frame}{}
  \Large
  About me:

  \normalsize
  \begin{itemize}
    \item Git user since 2010
    \item Worked with 100+ developers on the same code base for multiple years
    \item My favorite methodology is Cookie Driven Development
  \end{itemize}
\end{frame}

\section*{Setting the stage}

\begin{frame}[fragile]{}
  \begin{center}
    \Huge \mintinline{text}{git bisect}
  \end{center}
\end{frame}

\begin{frame}{}
  \begin{center}
    \LARGE Relies on good commit history
  \end{center}
\end{frame}

\section*{Commit history}

\begin{frame}{}
  \begin{center}
    \Huge What is a commit?
  \end{center}
  \pause

  \large
  \begin{itemize}
    \item A way of saving your work
    \pause
    \item A snapshot of a previous state
    \pause
    \item A way to document development over time
  \end{itemize}
\end{frame}

\begin{frame}{}
  \begin{center}
    \Huge Why is history important?
  \end{center}
\end{frame}

\begin{frame}{}
  \begin{center}
    \Huge Because it tells us what has changed
  \end{center}
\end{frame}

\begin{frame}[fragile]{}
  \begin{minted}{bash}
    d690bbc add content to foo
    85c6bfe fix typo
    b4a6944 WIP: add content to foo
  \end{minted}

  \center \Large vs.

  \normalsize
  \begin{minted}{bash}
    0a5e89d add content to foo
  \end{minted}
\end{frame}

\begin{frame}{}
  \LARGE Not all history is worth keeping
\end{frame}

\begin{frame}{}
  \LARGE Not every commit should be saved
\end{frame}

\section*{Cleaning up history}

\begin{frame}{}
  \begin{center}
    \huge \mintinline{text}{git commit --amend}
  \end{center}
\end{frame}

\begin{frame}{}
  \begin{center}
    \huge Lets you edit the last commit
  \end{center}
\end{frame}

\begin{frame}{}
  \begin{center}
    \huge But what if that's not enough?
  \end{center}
\end{frame}

\begin{frame}{}
  \begin{center}
    \huge \mintinline{text}{git rebase --interactive}
  \end{center}
\end{frame}

\begin{frame}[fragile]{}
  \Large From the manual for \mintinline{text}{git-rebase(1)}:

  \large
  \begin{minted}{text}
    The interactive mode is meant for this type of
    workflow:

      1. have a wonderful idea
      2. hack on the code
      3. prepare a series for submission
      4. submit
  \end{minted}
\end{frame}

\begin{frame}[fragile]{}
  \Large
  Our original history:

  \begin{minted}{bash}
    885bb69 add content to bar
    d690bbc add content to foo
    85c6bfe fix typo
    b4a6944 WIP: add content to foo
    bb3be90 initial commit
  \end{minted}
\end{frame}

\begin{frame}[fragile]{}
  \Large
  Specify starting point:

  \begin{minted}{bash}
    $ git rebase -i bb3be90 # initial commit
  \end{minted}
  \pause

  \normalsize
  This lets you edit the list of commits in your editor.
  \pause

  (Pro tip: always have an initial commit you can use as a starting point.)
\end{frame}

\begin{frame}[fragile]{}
  \Large
  We decide what should be done with the commits:

  \begin{minted}{bash}
    reword b4a6944 WIP: add content to foo
    fixup 85c6bfe fix typo
    fixup d690bbc add content to foo
    pick 885bb69 add content to bar
  \end{minted}
\end{frame}

\begin{frame}[fragile]{}
  \Large
  Our new history looks like this:

  \begin{minted}{bash}
    6a38155 add content to bar
    1436cf1 add content to foo
    bb3be90 initial commit
  \end{minted}

  \large
  (Note that the hashes have changed.)

\end{frame}

\begin{frame}[fragile]{}
  \Large
  You will have to tell Git to overwrite the remote:

  \begin{minted}{bash}
    $ git push --force
  \end{minted}

  \normalsize
  (Provided that the commits we rebased had been pushed earlier.)
\end{frame}

\begin{frame}{}
  \LARGE
  Use it to clean up branches before merging
\end{frame}

\begin{frame}{}
  \begin{alertblock}{\Huge NOTE}
    \Large
    Make sure you coordinate with your collaborators
  \end{alertblock}
\end{frame}

\begin{frame}{}
  \LARGE
  Rewriting history can lead to trouble for others
  \pause

  \normalsize
  A branch containing commits that no longer exist on the remote will
  result in nasty conflicts.
\end{frame}

\begin{frame}{}
  \Huge Coordination is the key
\end{frame}

\begin{frame}[fragile]{}
  \LARGE
  Don't use \mintinline{text}{rebase} for "public" commits
  \pause

  \large
  (E.g. \mintinline{text}{master} in a public repo or a branch that
  many are based on)
\end{frame}

\section*{Recovering from a history rewrite}

\begin{frame}{}
  \Large
  Our collaborator has rebased the branch we are working on
  \pause

  We have unpushed commits in that branch
\end{frame}

\begin{frame}[fragile]{}
  \Large
  Our branch:

  \begin{minted}{bash}
    819defb add README
    885bb69 add content to bar
    d690bbc add content to foo
    85c6bfe fix typo
    b4a6944 WIP: add content to foo
    bb3be90 initial commit
  \end{minted}
\end{frame}

\begin{frame}[fragile]{}
  \Large
  The remote branch:

  \begin{minted}{bash}
    6a38155 add content to bar
    1436cf1 add content to foo
    bb3be90 initial commit
  \end{minted}
\end{frame}

\begin{frame}[fragile]{}
  \Large
  Create a temp branch based on your local branch:

  \begin{minted}{bash}
    $ git checkout -b add-foo-and-bar-temp
  \end{minted}
\end{frame}

\begin{frame}[fragile]{}
  \Large
  Get your local branch up to date with the remote:

  \begin{minted}{bash}
    $ git branch -D add-foo-and-bar
    $ git checkout -t origin/add-foo-and-bar
  \end{minted}
  \pause

  Or

  \begin{minted}{bash}
    $ git checkout add-foo-and-bar
    # find out how many commits differ
    $ git fetch --all && git status
    $ git reset --hard HEAD~n
    $ git pull
  \end{minted}
  \pause

  Or

  \begin{minted}{bash}
    $ git checkout add-foo-and-bar
    $ git reset --hard origin/add-foo-and-bar
  \end{minted}
\end{frame}

\begin{frame}[fragile]{}
  \Large
  Pick your unpushed commits:

  \begin{minted}{bash}
    # on branch add-foo-and-bar
    # get commit hashes
    $ git log add-foo-and-bar-temp
    $ git cherry-pick 819defb
  \end{minted}

  \normalsize
  (I tend to only cherry-pick individual commits.)
\end{frame}

\section*{Other flavors of rebase}

\begin{frame}[fragile]{}
  \Large
  Rant: \mintinline{text}{git pull} should never result in a merge
\end{frame}

\begin{frame}[fragile]{}
  \Large
  Tell Git to only apply remote changes if they can be fast-forwarded:

  \begin{center}
    \mintinline{text}{git config --global merge.ff only}
  \end{center}
\end{frame}

\begin{frame}[fragile]{}
  \Large
  Pull in remote changes and apply your local commits after:

  \begin{center}
    \mintinline{text}{git pull --rebase}
  \end{center}
\end{frame}

\begin{frame}[fragile]{}
  \LARGE Home assignment:

  \Large
  How can we use \mintinline{text}{git pull --rebase}
  to recover from a history rewrite?
  \pause

  \normalsize
  (Tip: \mintinline{text}{git rebase --interactive} is a good start.)
  \pause

  Bonus question: When will this not work?
\end{frame}

\begin{frame}[fragile]{}
  \Large
  Use \mintinline{text}{git rebase} to keep up to date:

  \begin{minted}{bash}
    $ git checkout master && git pull
    $ git checkout -b my-feature-branch
    # do stuff: add/commit/push
    # prepare for pull request
    $ git checkout master && git pull
    $ git checkout my-feature-branch
    # add the new changes to your branch
    $ git rebase master
  \end{minted}
\end{frame}

\section*{When things go horribly wrong}

\begin{frame}{}
  \begin{center}
    \Huge \mintinline{text}{git reflog}
  \end{center}
\end{frame}

\begin{frame}[fragile]{}
  \Large
  \mintinline{text}{git reflog} tracks changes to
  \mintinline{text}{HEAD} locally:

  \begin{minted}[fontsize=\tiny]{bash}
    3783c49 HEAD@{0}: cherry-pick: completely rewrite bar
    d5b962e HEAD@{1}: reset: moving to HEAD~1
    ac48e40 HEAD@{2}: rebase -i (finish): returning to refs/heads/master
    ac48e40 HEAD@{3}: rebase -i (pick): completely rewrite bar
    d5b962e HEAD@{4}: rebase -i (pick): add content to bar
    97e39a3 HEAD@{5}: rebase -i (fixup): add content to foo
    28a4547 HEAD@{6}: rebase -i (fixup): # This is a combination of 2 commits.
    d47e71c HEAD@{7}: rebase -i (reword): add content to foo
    b4a6944 HEAD@{8}: cherry-pick: fast-forward
    bb3be90 HEAD@{9}: rebase -i (start): checkout bb3be9022765add167e4d014479a8104d5c7db79
    819defb HEAD@{10}: checkout: moving from foo to master
  \end{minted}
\end{frame}

\begin{frame}[fragile]{}
  \Large
  Using \mintinline{text}{git reflog} we can find a point to step back
  to:

  \begin{minted}{bash}
    $ git rebase -i <commit>
    # here things go horribly wrong
    $ git reflog # to find a point before our mistake
    $ git reset --hard <another commit>
  \end{minted}
\end{frame}

\begin{frame}{}
  \begin{alertblock}{\Huge NOTE}
    Careful when deleting branches, \mintinline{text}{git reflog} will
    not help you there.
  \end{alertblock}
\end{frame}

\section*{Recap}

\begin{frame}{}
  \Large
  Git commands to make your life easier:

  \normalsize
  \begin{itemize}
    \item \mintinline{text}{git rebase} to ensure you have a clean history
    \item \mintinline{text}{git cherry-pick} to pick (individual) commits
    \item \mintinline{text}{git reflog} to recover from mishaps
  \end{itemize}
\end{frame}

\begin{frame}{}
  \begin{center}
    \Huge
    Communication is essential
  \end{center}
\end{frame}

\begin{frame}[standout]{}
  \begin{center}
    \Huge
    Questions?
  \end{center}
\end{frame}

\begin{frame}[standout]{}
  \begin{center}
    This work is licensed under Creative Commons Attribution-ShareAlike 4.0 International.
    \url{https://creativecommons.org/licenses/by-sa/4.0/}
  \end{center}
\end{frame}

\end{document}
